\documentclass[a4paper,14pt]{extreport}

\usepackage{mmap}
\usepackage[T2A]{fontenc}
\usepackage[utf8]{inputenc}
\usepackage[english,russian]{babel}

\renewcommand{\ttdefault}{PTMono-TLF}

% Текст отчёта следует печатать, соблюдая следующие размеры полей:
% левое — 30 мм, правое — 15 мм, верхнее и нижнее — 20 мм.
\usepackage[left=20mm, right=15mm, top=15mm, bottom=15mm]{geometry}

% \setlength{\parindent}{1.25cm} % Абзацный отступ

\usepackage{setspace}
\onehalfspacing % Полуторный интервал

\frenchspacing % Равномерные пробелы
\usepackage{indentfirst} % Красная строка

\usepackage{microtype}
\sloppy

\usepackage{titlesec}
\titlespacing*{\chapter}{0pt}{-30pt}{8pt}
\titlespacing*{\section}{\parindent}{*4}{*4}
\titlespacing*{\subsection}{\parindent}{*4}{*4}
\titleformat{\chapter}{\LARGE\bfseries}{\thechapter}{20pt}{\LARGE\bfseries}
\titleformat{\section}{\Large\bfseries}{\thesection}{40pt}{\Large\bfseries}

\usepackage{graphicx}
\usepackage{caption}
\usepackage{float}

\usepackage[unicode,pdftex]{hyperref}
\hypersetup{
	hidelinks=true,
	colorlinks=true,
	linkcolor=black,
	urlcolor=blue,
}

%% title begin
\usepackage{wrapfig}

\makeatletter
	\def\vhrulefill#1{\leavevmode\leaders\hrule\@height#1\hfill \kern\z@}
\makeatother
%% title end

%% begin code
\usepackage{listings}
\usepackage{xcolor}

\lstset{
	basicstyle=\scriptsize\ttfamily,
	breakatwhitespace=true,
	breaklines=true,
	commentstyle=\color{gray},
	frame=single,
	keywordstyle=\color{blue},
	numbers=left,
	numbersep=5pt,
	numberstyle=\tiny\ttfamily\color{gray},
	showstringspaces=false,
	stringstyle=\color{red},
	tabsize=8
}

\newcommand{\code}[1]{\texttt{#1}}
%% end code

\usepackage{amsmath}
\usepackage{amssymb}
\usepackage{commath}
\usepackage{icomma}


\begin{document}

\begin{titlepage}
	\centering

	\begin{wrapfigure}[7]{l}{0.14\linewidth}
		\vspace{5mm}
		\hspace{-5.8mm}
		\includegraphics[width=0.93\linewidth]{inc/img/bmstu-logo}
	\end{wrapfigure}
	{\singlespacing \footnotesize \bfseries Министерство науки и высшего образования Российской Федерации\\Федеральное государственное бюджетное образовательное учреждение\\высшего образования\\<<Московский государственный технический университет\\имени Н.~Э.~Баумана\\ (национальный исследовательский университет)>>\\(МГТУ им. Н.~Э.~Баумана)\\}

	\vspace{-2.2mm}
	\vhrulefill{0.9mm}\\
	\vspace{-7mm}
	\vhrulefill{0.2mm}\\
	\vspace{2mm}

	{\doublespacing \small \raggedright ФАКУЛЬТЕТ \hspace{25mm} «Информатика и системы управления»\\
	КАФЕДРА \hspace{5mm} «Программное обеспечение ЭВМ и информационные технологии»\\}

	\vspace{30mm}

	\textbf{ОТЧЁТ}\\
	По лабораторной работе №1\\
	По курсу: «Моделирование»\\
	Тема: «ОДУ. Задача Коши. Приближённый аналитический метод Пикара и численный метод Эйлера»\\

	\vspace{60mm}

	\hspace{70mm} Студент:       \hfill Керимов~А.~Ш.\\
	\hspace{70mm} Группа:        \hfill ИУ7-64Б\\
	\hspace{70mm} Преподаватель: \hfill Градов~В.~М.\\
	{\raggedright \hspace{70mm} Оценка: \hfill \hrulefill\\}

	\vfill

	Москва\\
	\the\year
\end{titlepage}


Рассмотрим задачу с начальным условием для дифференциального уравнения (задачу Коши):
\begin{equation}
	\left\{
	\begin{aligned}
		u'(x) &= f(x, u), \\
		u(\xi) &= \eta
	\end{aligned}
	\right.
\end{equation}

Решение можно найти приближённым аналитическим методом Пикара:
\begin{equation}
	\begin{aligned}
		y^{(0)}(x) &= \eta \\
		y^{(n+1)}(x) &= \eta + \int\limits_{\xi}^x f(t, y^{(n)}(t))\,\mathrm dt \\
	\end{aligned}
\end{equation}

На примере $u'(x) = x^2 + y^2$ при $u(0) = 0$:
\begin{equation}
	\begin{aligned}
		y^{(0)}(0) &= 0, \\
		y^{(1)}(x) &= 0 + \int_0^x t^2\,\mathrm dt = \frac{x^3}{3}, \\
		y^{(2)}(x) &= 0 + \int_0^x \bigg[t^2 + \bigg(\frac{t^3}{3}\bigg)^2\bigg] \,\mathrm dt = \frac{x^3}{3} + \frac{x^7}{63}, \\
		y^{(3)}(x) &= 0 + \int_0^x \bigg[t^2 + \bigg(\frac{t^3}{3} + \frac{t^7}{63}\bigg)^2\bigg] \,\mathrm dt = \frac{x^3}{3} + \frac{x^7}{63} + \frac{2x^{11}}{2079} + \frac{x^{15}}{59535}, \\
		\dots
	\end{aligned}
\end{equation}

Кроме того, эту задачу можно решить численными методами.
Следующая формула для явного способа:
\begin{equation}
	y_{n+1} = y_n + h \cdot f(x_n, y_n)
\end{equation}

Похожим образом выглядит неявный метод:
\begin{equation}
	y_{n+1} = y_n + h \cdot f(x_{n+1}, y_{n+1})
\end{equation}

Стоит заметить, что для всех рассмотренных методов результат будет тем лучше, чем ближе значение $x$ к $\xi$.

Реализованная программа производит расчет для $f(x, y) = x^2 + y^2$.
Исходя из этого, можно упростить нахождения решения в неявном виде: получаем квадратное уравнение и в качестве решения берем меньший корень.

\lstinputlisting[language=C++, linerange={9-30}, caption={Реализация аналитического метода Пикара}]{../cauchy/cauchy.cpp}

\lstinputlisting[language=C++, linerange={32-45}, caption={Реализация явного численного метода}]{../cauchy/cauchy.cpp}

\lstinputlisting[language=C++, linerange={47-76}, caption={Реализация неявного численного метода}]{../cauchy/cauchy.cpp}

\end{document}
