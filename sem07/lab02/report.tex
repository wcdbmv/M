\documentclass[a4paper,oneside,12pt]{extreport}

\usepackage{mmap}
\usepackage[T2A]{fontenc}
\usepackage[utf8]{inputenc}
\usepackage[english,russian]{babel}

\renewcommand{\ttdefault}{PTMono-TLF}

% Текст отчёта следует печатать, соблюдая следующие размеры полей:
% левое — 30 мм, правое — 15 мм, верхнее и нижнее — 20 мм.
\usepackage[left=20mm, right=15mm, top=15mm, bottom=15mm]{geometry}

% \setlength{\parindent}{1.25cm} % Абзацный отступ

\usepackage{setspace}
\onehalfspacing % Полуторный интервал

\frenchspacing % Равномерные пробелы
\usepackage{indentfirst} % Красная строка

\usepackage{microtype}
\sloppy

\usepackage{titlesec}
\titlespacing*{\chapter}{0pt}{-30pt}{8pt}
\titlespacing*{\section}{\parindent}{*4}{*4}
\titlespacing*{\subsection}{\parindent}{*4}{*4}
\titleformat{\chapter}{\LARGE\bfseries}{\thechapter}{20pt}{\LARGE\bfseries}
\titleformat{\section}{\Large\bfseries}{\thesection}{40pt}{\Large\bfseries}

\usepackage{graphicx}
\usepackage{caption}
\usepackage{float}

\usepackage[unicode,pdftex]{hyperref}
\hypersetup{
	hidelinks=true,
	colorlinks=true,
	linkcolor=black,
	urlcolor=blue,
}

%% title begin
\usepackage{wrapfig}

\makeatletter
	\def\vhrulefill#1{\leavevmode\leaders\hrule\@height#1\hfill \kern\z@}
\makeatother
%% title end

%% begin code
\usepackage{listings}
\usepackage{xcolor}

\lstset{
	basicstyle=\scriptsize\ttfamily,
	breakatwhitespace=true,
	breaklines=true,
	commentstyle=\color{gray},
	frame=single,
	keywordstyle=\color{blue},
	numbers=left,
	numbersep=5pt,
	numberstyle=\tiny\ttfamily\color{gray},
	showstringspaces=false,
	stringstyle=\color{red},
	tabsize=8
}

\newcommand{\code}[1]{\texttt{#1}}
%% end code

\usepackage{amsmath}
\usepackage{amssymb}
\usepackage{commath}
\usepackage{icomma}


\begin{document}

\begin{titlepage}
	\centering

	\begin{wrapfigure}[7]{l}{0.14\linewidth}
		\vspace{5mm}
		\hspace{-5.8mm}
		\includegraphics[width=0.93\linewidth]{inc/img/bmstu-logo}
	\end{wrapfigure}
	{\singlespacing \footnotesize \bfseries Министерство науки и высшего образования Российской Федерации\\Федеральное государственное бюджетное образовательное учреждение\\высшего образования\\<<Московский государственный технический университет\\имени Н.~Э.~Баумана\\ (национальный исследовательский университет)>>\\(МГТУ им. Н.~Э.~Баумана)\\}

	\vspace{-2.2mm}
	\vhrulefill{0.9mm}\\
	\vspace{-7mm}
	\vhrulefill{0.2mm}\\
	\vspace{2mm}

	{\doublespacing \small \raggedright ФАКУЛЬТЕТ \hspace{25mm} «Информатика и системы управления»\\
	КАФЕДРА \hspace{5mm} «Программное обеспечение ЭВМ и информационные технологии»\\}

	\vspace{30mm}

	\textbf{ОТЧЁТ}\\
	По лабораторной работе №1\\
	По курсу: «Моделирование»\\
	Тема: «ОДУ. Задача Коши. Приближённый аналитический метод Пикара и численный метод Эйлера»\\

	\vspace{60mm}

	\hspace{70mm} Студент:       \hfill Керимов~А.~Ш.\\
	\hspace{70mm} Группа:        \hfill ИУ7-64Б\\
	\hspace{70mm} Преподаватель: \hfill Градов~В.~М.\\
	{\raggedright \hspace{70mm} Оценка: \hfill \hrulefill\\}

	\vfill

	Москва\\
	\the\year
\end{titlepage}


\tableofcontents

\chapter{Теоретическая часть}

\section{Равномерное распределение}

Случайная величина имеет равномерное распределение на отрезке $[a, b]$, где $a, b \in \mathbb R$, если её плотность имеет вид:
\begin{equation}
	f_X(x) =
	\left\{
	\begin{aligned}
		& \frac{1}{b - a}, & x \in [a, b], \\
		& 0,               & x \notin [a, b]. \\
	\end{aligned}
	\right.
\end{equation}

Функция равномерного распределения:
\begin{equation}
	F_X(x) =
	\left\{
	\begin{aligned}
		& 0,                   & x < a, \\
		& \frac{x - a}{b - a}, & a \leqslant x < b, \\
		& 1,                   & x \geqslant b. \\
	\end{aligned}
	\right.
\end{equation}

Обозначают: $X \sim R(a, b)$.

\section{Нормальное распределение}

Случайная величина имеет нормальное распределение, если её плотность имеет вид:
\begin{equation}
	f_X(x) = \frac{1}{\sigma\sqrt{2\pi}}e^{-\frac{(x - \mu)^2}{2\sigma^2}},
\end{equation}
где
\begin{itemize}
	\item параметр $\mu \in \mathbb R$ — математическое ожидание, определяет центр симметрии распределения,
	\item параметр $\sigma \in \mathbb R_{> 0}$ — среднеквадратичное отклонение, определяет степень разброса случайной величины относительно математического ожидания.
\end{itemize}

Функция нормального распределения:
\begin{equation}
	F_X(x) = \frac{1}{\sigma\sqrt{2\pi}}\int\limits_{-\infty}^{x} e^{-\frac{(x - \mu)^2}{2\sigma^2}} \,\mathrm dx,
\end{equation}

Обозначают: $X \sim N(\mu, \sigma^2)$.

\chapter{Результат работы}

\section{Равномерное распределение}

\begin{figure}[H]
	\centering
	\begin{tikzpicture}
		\begin{axis}[
			samples = 200,
			const plot mark mid,
			domain = -1.5:3.5,
			ymin = 0,
		]
			\addplot [very thick, orange] {unipdf(0, 1)};
			\addlegendentry{$a = 0; b = 1$};

			\addplot [very thick, cyan] {unipdf(0.5, 2.5)};
			\addlegendentry{$a = 0,5; b = 2,5$};

			\addplot [very thick, teal] {unipdf(-1, 3)};
			\addlegendentry{$a = -1; b = 3$};
		\end{axis}
	\end{tikzpicture}
	\caption{Графики плотности равномерного распределения при различных $a$ и $b$}
\end{figure}

\begin{figure}[H]
	\centering
	\begin{tikzpicture}
		\begin{axis}[
			samples = 200,
			mark = none,
			domain = -1.5:3.5,
			ymin = 0,
		]
			\addplot [very thick, orange] {unicdf(0, 1)};
			\addlegendentry{$a = 0; b = 1$};

			\addplot [very thick, cyan] {unicdf(0.5, 2.5)};
			\addlegendentry{$a = 0,5; b = 2,5$};

			\addplot [very thick, teal] {unicdf(-1, 3)};
			\addlegendentry{$a = -1; b = 3$};
		\end{axis}
	\end{tikzpicture}
	\caption{Графики функции равномерного распределения при различных $a$ и $b$}
\end{figure}

\section{Нормальное распределение}

\begin{figure}[H]
	\centering
	\begin{tikzpicture}
		\begin{axis}[
			samples = 200,
			mark=none,
			smooth,
			domain = -5:5,
			ymin = 0,
		]
			\addplot [very thick, orange] {normpdf(0, 1)};
			\addlegendentry{$\mu = 0; \sigma^2 = 1$};

			\addplot [very thick, cyan] {normpdf(0, sqrt(5))};
			\addlegendentry{$\mu = 0; \sigma^2 = 5$};

			\addplot [very thick, teal] {normpdf(0, sqrt(0.2))};
			\addlegendentry{$\mu = 0; \sigma^2 = 0,2$};

			\addplot [very thick, violet] {normpdf(2, 0.5)};
			\addlegendentry{$\mu = 2; \sigma^2 = 0,5$};
		\end{axis}
	\end{tikzpicture}
	\caption{Графики плотности нормального распределения при различных $\mu$ и $\sigma^2$}
\end{figure}

\begin{figure}[H]
	\centering
	\begin{tikzpicture}
		\begin{axis}[
			samples = 200,
			mark=none,
			smooth,
			domain = -5:5,
			ymin = 0,
		]
			\addplot [very thick, orange] {normcdf(0, 1)};
			\addlegendentry{$\mu = 0; \sigma^2 = 1$};

			\addplot [very thick, cyan] {normcdf(0, sqrt(5))};
			\addlegendentry{$\mu = 0; \sigma^2 = 5$};

			\addplot [very thick, teal] {normcdf(0, sqrt(0.2))};
			\addlegendentry{$\mu = 0; \sigma^2 = 0,2$};

			\addplot [very thick, violet] {normcdf(2, sqrt(0.5))};
			\addlegendentry{$\mu = 2; \sigma^2 = 0,5$};
		\end{axis}
	\end{tikzpicture}
	\caption{Графики функции нормального распределения при различных $\mu$ и $\sigma^2$}
\end{figure}

\chapter*{Вывод}
\addcontentsline{toc}{chapter}{Вывод}

В ходе выполнения лабораторной работы были рассмотрены равномерное и нормальное распределения, а также построены графики функций и плотностей распределений при различных параметрах.

\end{document}
