\documentclass[a4paper,oneside,12pt]{extreport}

\usepackage{mmap}
\usepackage[T2A]{fontenc}
\usepackage[utf8]{inputenc}
\usepackage[english,russian]{babel}

\renewcommand{\ttdefault}{PTMono-TLF}

% Текст отчёта следует печатать, соблюдая следующие размеры полей:
% левое — 30 мм, правое — 15 мм, верхнее и нижнее — 20 мм.
\usepackage[left=20mm, right=15mm, top=15mm, bottom=15mm]{geometry}

% \setlength{\parindent}{1.25cm} % Абзацный отступ

\usepackage{setspace}
\onehalfspacing % Полуторный интервал

\frenchspacing % Равномерные пробелы
\usepackage{indentfirst} % Красная строка

\usepackage{microtype}
\sloppy

\usepackage{titlesec}
\titlespacing*{\chapter}{0pt}{-30pt}{8pt}
\titlespacing*{\section}{\parindent}{*4}{*4}
\titlespacing*{\subsection}{\parindent}{*4}{*4}
\titleformat{\chapter}{\LARGE\bfseries}{\thechapter}{20pt}{\LARGE\bfseries}
\titleformat{\section}{\Large\bfseries}{\thesection}{40pt}{\Large\bfseries}

\usepackage{graphicx}
\usepackage{caption}
\usepackage{float}

\usepackage[unicode,pdftex]{hyperref}
\hypersetup{
	hidelinks=true,
	colorlinks=true,
	linkcolor=black,
	urlcolor=blue,
}

%% title begin
\usepackage{wrapfig}

\makeatletter
	\def\vhrulefill#1{\leavevmode\leaders\hrule\@height#1\hfill \kern\z@}
\makeatother
%% title end

%% begin code
\usepackage{listings}
\usepackage{xcolor}

\lstset{
	basicstyle=\scriptsize\ttfamily,
	breakatwhitespace=true,
	breaklines=true,
	commentstyle=\color{gray},
	frame=single,
	keywordstyle=\color{blue},
	numbers=left,
	numbersep=5pt,
	numberstyle=\tiny\ttfamily\color{gray},
	showstringspaces=false,
	stringstyle=\color{red},
	tabsize=8
}

\newcommand{\code}[1]{\texttt{#1}}
%% end code

\usepackage{amsmath}
\usepackage{amssymb}
\usepackage{commath}
\usepackage{icomma}


\begin{document}

\begin{titlepage}
	\centering

	\begin{wrapfigure}[7]{l}{0.14\linewidth}
		\vspace{5mm}
		\hspace{-5.8mm}
		\includegraphics[width=0.93\linewidth]{inc/img/bmstu-logo}
	\end{wrapfigure}
	{\singlespacing \footnotesize \bfseries Министерство науки и высшего образования Российской Федерации\\Федеральное государственное бюджетное образовательное учреждение\\высшего образования\\<<Московский государственный технический университет\\имени Н.~Э.~Баумана\\ (национальный исследовательский университет)>>\\(МГТУ им. Н.~Э.~Баумана)\\}

	\vspace{-2.2mm}
	\vhrulefill{0.9mm}\\
	\vspace{-7mm}
	\vhrulefill{0.2mm}\\
	\vspace{2mm}

	{\doublespacing \small \raggedright ФАКУЛЬТЕТ \hspace{25mm} «Информатика и системы управления»\\
	КАФЕДРА \hspace{5mm} «Программное обеспечение ЭВМ и информационные технологии»\\}

	\vspace{30mm}

	\textbf{ОТЧЁТ}\\
	По лабораторной работе №1\\
	По курсу: «Моделирование»\\
	Тема: «ОДУ. Задача Коши. Приближённый аналитический метод Пикара и численный метод Эйлера»\\

	\vspace{60mm}

	\hspace{70mm} Студент:       \hfill Керимов~А.~Ш.\\
	\hspace{70mm} Группа:        \hfill ИУ7-64Б\\
	\hspace{70mm} Преподаватель: \hfill Градов~В.~М.\\
	{\raggedright \hspace{70mm} Оценка: \hfill \hrulefill\\}

	\vfill

	Москва\\
	\the\year
\end{titlepage}


\tableofcontents

\chapter{Формализация}

\section{Задание}

В Люберецкий МФЦ приходят клиенты через интервал времени $4 \pm 2$ минуты и становятся в очередь к терминалу по получению талонов.
На получение талона в среднем у клиента уходит $3 \pm 1$ минуты.
Далее равновероятно клиенты становятся в одну из трёх очередей за услугами типа 1, 2 или 3: оформление документов, получение документов, остальные услуги.
На обслуживание клиентов в каждом окне соответственно уходит $15 \pm 5$, $10 \pm 2$, $30 \pm 10$ минут.
С вероятностью $p_{\text{возвр}} = 0,33$ клиент после получения услуги типа 1 становится обратно в очередь для получения талона на следующую услугу.
Максимальная длина $L$ любой очереди в условиях пандемии — 5 человек.
Промоделировать процесс обработки 1000 клиентов.

\section{Концептульная модель}

\begin{figure}[H]
	\centering
	\includegraphics[width=0.9\linewidth]{inc/img/conceptual-model.pdf}
	\caption{Концептуальная модель}
	\label{conceptual-model-qt}
\end{figure}

В процессе взаимодействия клиентов с Люберецким МФЦ возможно:
\begin{itemize}
	\item Режим нормального обслуживания, т. е. клиент получил услугу.
	\item Режим отказа в обслуживании клиента, когда очередь, в которую становится клиент, заполнена.
\end{itemize}

\chapter{Листинг}

\lstinputlisting[language=GPSS]{inc/src/main.gpss}

\chapter{Результат работы}

\begin{figure}[H]
	\centering
	\includegraphics[width=\linewidth]{inc/img/result-values.jpg}
	\caption{Сохраняемые величины}
\end{figure}

\begin{figure}[H]
	\centering
	\includegraphics[width=\linewidth]{inc/img/result-devices.jpg}
	\caption{Устройства}
\end{figure}

\begin{figure}[H]
	\centering
	\includegraphics[width=\linewidth]{inc/img/result-queues.jpg}
	\caption{Очереди}
\end{figure}

\chapter*{Вывод}
\addcontentsline{toc}{chapter}{Вывод}

Разработана программа в системе GPSS, предназначенная для моделирования процесса обработки клиентов в Люберецком МФЦ, позволяющая определить количество потерянных заявок и вероятность отказа в обслуживании.

\end{document}
